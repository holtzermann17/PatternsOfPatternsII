\title{Patterns of Patterns II}

\author{Joseph Corneli}
\authornote{Corresponding author, jcorneli@brookes.ac.uk.}
\email{jcorneli@brookes.ac.uk}
\orcid{1234-5678-9012}
\affiliation{%
  \institution{Oxford Brookes University}
  \streetaddress{Gipsy Lane}
  \city{Oxford}
  \country{UK}
  \postcode{OX3 0BP}
}
\affiliation{%
  \institution{Hyperreal Enterprises Ltd}
  \streetaddress{81 St Clement’s St}
  \city{Oxford}
  \country{UK}
  \postcode{OX4 1AW}}

\author{Noorah Alhasan}
\author{Leo Vivier}
\author{Alex Murphy}
\author{Raymond S. Puzio}
%\authornotemark[1]
\email{rsp@hyperreal.enterprises}
\affiliation{%
  \institution{Hyperreal Enterprises Ltd}
  \streetaddress{81 St Clement’s St}
  \city{Oxford}
  \country{UK}
  \postcode{OX4 1AW}}

\author{Abby Tabor}
\affiliation{%
  \institution{University of the West of England}
  \streetaddress{Faculty of Health and Applied Sciences (HAS), Frenchay Campus, Coldharbour Lane}
  \city{Bristol}
  \state{England}
  \country{UK}
  \postcode{BS16 1QY}}
\email{abby.tabor@uwe.ac.uk}

%% \author{Vitor Bruno}
%% \affiliation{%
%%   \institution{Milestone English}
%%   \streetaddress{Rua Trieste 170, ap2}
%%   \city{Palhoca}
%%   \state{SC}
%%   \country{Brazil}
%%   \postcode{88132-227}}
%% \email{chief@milestoneenglishcourse.com}

%% % \author{Paola Ricaurte}
%% % \affiliation{%
%% %   \institution{Tecnologico de Monterrey}
%% %   \streetaddress{Calle del Puente 222 Col. Ejidos de Huipulco, Tlalpan}
%% %   \city{Mexico City}
%% %   \country{Mexico}
%% % }
%% % \email{pricaurt@tec.mx}

\author{Sridevi Ayloo}
\affiliation{%
  \institution{New York City College of Technology}
  \streetaddress{300 Jay St}
  \city{Brooklyn}
  \postcode{11201}
  \country{USA}
}
\email{pricaurt@tec.mx}

\author{Charlotte Pierce}
\affiliation{%
  \institution{Pierce Press}
  \streetaddress{PO Box 206}
  \city{Arlington MA}
  \country{USA}
  \postcode{02476}
}
\email{charlotte.pierce@gmail.com}

\author{Mary Tedeschi}
\author{Manvinder Singh}
\author{Kajol Khetan}
\affiliation{%
  \institution{Baruch College}
 \streetaddress{PO Box 802738}
 \city{New York}
 \state{NY}
  \country{USA}
  \postcode{60680}}
\email{mtedeschi@pace.edu}

\author{Charles J. Danoff}
\affiliation{%
  \institution{Mr Danoff’s Teaching Laboratory}
 \streetaddress{PO Box 802738}
 \city{Chicago}
 \state{IL}
  \country{USA}
  \postcode{60680}}
\email{contact@mr.danoff.org}

%%
%% By default, the full list of authors will be used in the page
%% headers. Often, this list is too long, and will overlap
%% other information printed in the page headers. This command allows
%% the author to define a more concise list
%% of authors' names for this purpose.
\renewcommand{\shortauthors}{Corneli et al.}


% (1) Review the intention: what do we expect to learn or make together?

% - Joe: Wanted to walk through the PoP paper with Rebecca, in order to help solidify my own grasp of the concepts and get her feedback.
% - Rebecca: This topic sounded interesting when you mentioned it and I wanted to learn about Pop

% (2) Establish what is happening: what and how are we learning?

% - Indeed we did speed through the paper, Rebecca pointed out a few places where there was friction with the wording or concepts, like ``PEER-TO-PEER'' and also suggested Operational Research and Strategy as an appropriate topic; mentioned “improvisatory” style
% - Interruptions were welcomed!
% - Rebecca: was hesitant to interrupt the narrative
% - This is a bit different from usual IEAI style...

% (3) What are some different perspectives on what’s happening?

% - Joe: appreciate the time Rebecca has put into this a lot, and I also think this was a good way to present the paper
% - Rebecca: I think you assume knowledge in the presentation, and I think you need to assume the listener (if they aren’t in the area) that they don’t know anything.  It wouldn’t be patronising to explain the basic concepts.

% (4) What did we learn or change?

% - Talking to Karl, he would reconise one of the areas (probably) but not necessarily the other two.  Everyone is going to be new to some of these concepts.
% - This was great as a ``final edit'' — we will also be able to edit this paper until December
% - RR: is it your aim to automate?

% (5) What else should we change going forward?

% - Joe: if it would be helpful to RR, I’d certainly be happy to meet again about these ideas
% - Would this (patterns of patterns) to actually be useful for ethical AI?
% - E.g., rethink in the context of moral machines

%%
%% The abstract is a short summary of the work to be presented in the
%% article.
% distributed peer-to-peer networks
\begin{abstract}
We review how our earlier theorization of pattern methods fares in the
wild.  The “wild” here included a graduate school classroom in New
York, a workshop at a transdisciplinary conference in Arizona, a
nascent citizen science project in Bristol, and a professional
development day for a university in Oxford.  We encountered unexpected
challenges such as working with students in a HyFlex classroom,
getting conference attendees to feel comfortable evaluating the
conference they were presently attending, and adapting our plans on
the fly when leading workshops with surprising attendee responses.  We
describe and refine pattern specifications that will help other
practitioners of patterns in their own forays into the wild.
\end{abstract}

%%
%% The code below is generated by the tool at http://dl.acm.org/ccs.cfm.
%% Please copy and paste the code instead of the example below.
%%
\begin{CCSXML}
<ccs2012>
<concept>
<concept_id>10003456</concept_id>
<concept_desc>Social and professional topics</concept_desc>
<concept_significance>500</concept_significance>
</concept>
<concept>
<concept_id>10011007.10011074.10011075</concept_id>
<concept_desc>Software and its engineering~Designing software</concept_desc>
<concept_significance>300</concept_significance>
</concept>
<concept>
<concept_id>10011007.10011074.10011134.10003559</concept_id>
<concept_desc>Software and its engineering~Open source model</concept_desc>
<concept_significance>300</concept_significance>
</concept>
<concept>
<concept_id>10010405.10010481</concept_id>
<concept_desc>Applied computing~Operations research</concept_desc>
<concept_significance>300</concept_significance>
</concept>
<concept>
<concept_id>10010147.10010341</concept_id>
<concept_desc>Computing methodologies~Modeling and simulation</concept_desc>
<concept_significance>100</concept_significance>
</concept>
</ccs2012>
\end{CCSXML}

\ccsdesc[500]{Social and professional topics}
\ccsdesc[300]{Software and its engineering~Designing software}
\ccsdesc[300]{Software and its engineering~Open source model}
\ccsdesc[300]{Applied computing~Operations research}
\ccsdesc[100]{Computing methodologies~Modeling and simulation}


%%
%% Keywords. The author(s) should pick words that accurately describe
%% the work being presented. Separate the keywords with commas.
\keywords{Design Patterns, Pattern Languages, Action Reviews, Futures
Studies, Causal Layered Analysis, Emacs, Free Software, Peeragogy,
Climate Change, Innovation, Anticipation}

%\authorsaddresses{This command processes the author and affiliation and title This command processes the author and affiliation and title This command processes the author and affiliation and title This command processes the author and affiliation and title}

%%
%% This command processes the author and affiliation and title
%% information and builds the first part of the formatted document.
\maketitle
